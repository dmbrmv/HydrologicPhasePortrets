\documentclass[11pt,a4paper]{article}

% Packages
\usepackage[utf8]{inputenc}
\usepackage{amsmath,amssymb,amsfonts}
\usepackage{graphicx}
\usepackage{geometry}
\geometry{left=2.5cm,right=2.5cm,top=2.5cm,bottom=2.5cm}
\usepackage{natbib}
\usepackage[colorlinks=true,linkcolor=blue,citecolor=blue]{hyperref}

\title{Identification of Dynamically Similar Hydrological Systems Using Singular Spectrum Analysis and Convergent Cross Mapping}

\author{[Your Name]$^{1,2}$ \\
\small $^{1}$Institution, Address, Country\\
\small $^{2}$Institution, Address, Country\\
\small Email: \texttt{your.email@example.com}}

\date{\today}

\begin{document}

\maketitle

\begin{abstract}
This study introduces an integrative approach combining Singular Spectrum Analysis (SSA) and Convergent Cross Mapping (CCM) to identify clusters of dynamically similar hydrological systems.
We use the Lipschitz coefficient, derived from reconstructed state-space trajectories, as a key indicator to quantify the predictability and nonlinear interactions among hydrological variables.
Our method offers a dynamic alternative to conventional classification schemes and can significantly enhance hydrological prediction and water resource management strategies. 
\end{abstract}

\section{Introduction}

Understanding the dynamics and interactions of hydrological systems is critical for accurate water resource management, flood forecasting, and sustainable environmental planning \citep{montanari2013understanding, sivapalan2012sociohydrology}.
Traditional hydrological classifications primarily rely on static catchment attributes such as land use, geology, and climatic averages, frequently neglecting nonlinear and dynamic characteristics inherent in catchment hydrology \citep{sawicz2011catchment, wagener2007catchment}.
Recent advances highlight the need for methods that explicitly consider dynamic and nonlinear behaviors in catchment similarity analysis \citep{wang2021nonlinear, krakovska2019causality}. 

Singular Spectrum Analysis (SSA) and Convergent Cross Mapping (CCM) offer robust approaches for the decomposition and analysis of complex hydrological time series.
SSA isolates meaningful components such as trends, periodicities, and noise \citep{ghil2002advanced, elsner2009singular}.
CCM reveals nonlinear causal interdependencies, facilitating the understanding of intricate hydrological relationships \citep{sugihara2012detecting, ye2015distinguishing}.
However, integrated applications of SSA and CCM remain limited in hydrological research, particularly for dynamic classification tasks.

In this paper, we propose combining SSA and CCM to cluster hydrological systems based on their dynamic similarity.
A distinctive feature of our method is the use of the Lipschitz coefficient—a type of Lyapunov exponent—calculated from reconstructed state-space trajectories.
This coefficient quantifies sensitivity to initial conditions and predictability, thus characterizing the intrinsic dynamics of hydrological catchments \citep{kantz2004nonlinear, wolf1985determining}.

Our hypothesis is that catchments grouped by similar Lipschitz coefficients will exhibit comparable hydrological dynamics, providing improved insights for modeling, forecasting, and management decisions.

\section{Materials and Methods}

\subsection{Study Area and Data}
- Description of hydrological catchments and available datasets.

\subsection{Singular Spectrum Analysis (SSA)}
- Theory and implementation details of SSA.

\subsection{Convergent Cross Mapping (CCM)}
- Theory and implementation details of CCM.

\subsection{Lipschitz Coefficient}
- Definition and calculation procedure.
- Application to hydrological time series.

\subsection{Clustering Procedure}
- Methodology to cluster catchments based on Lipschitz coefficients.

\section{Results}
\subsection{SSA Decomposition Results}
- Description and interpretation.

\subsection{CCM Causal Interaction Results}
- Visualization and analysis.

\subsection{Dynamic Clustering of Hydrological Systems}
- Clusters obtained, description, and interpretation.

\section{Discussion}
- Comparison with conventional hydrological classification methods.
- Implications for hydrological modeling and forecasting.

\section{Conclusions}
- Key findings and contribution to hydrological science.
- Recommendations for future research.

\section*{Acknowledgments}
- Funding, data provision, and collaboration acknowledgments.

\bibliographystyle{apalike}
\bibliography{references}

\end{document}